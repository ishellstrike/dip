\clearpage
\Large{Приложение А}
\begin{lstlisting}
public void RecalcNormals()
{
    for (int i = 0; i < Verteces.Count; i += 3)
    {
        var a = Verteces[i].Position;
        var b = Verteces[i + 1].Position;
        var c = Verteces[i + 2].Position;
        var normal = Vector3.Normalize(
              Vector3.Cross(c - a, b - a));
        var i0 = Verteces[i];
        var i1 = Verteces[i + 1];
        var i2 = Verteces[i + 2];

        i0.Normal = i1.Normal = i2.Normal = 
            normal;
        Verteces[i] = i0;
        Verteces[i + 1] = i1;
        Verteces[i + 2] = i2;
    }
    FlipNormals();
}
public void FlipNormals() {
    for (int i = 0; i < Verteces.Count; i++) {
        var vertex = Verteces[i];
        vertex.Normal *= -1;
        Verteces[i] = vertex;
    }
}
\end{lstlisting}


\clearpage
\Large{Приложение Б}
\begin{lstlisting}
__kernel void floatSquareDivPi(
    __global float * vx, 
    __global float * vy, 
    __global float * vz,
    __global float * nx, 
    __global float * ny, 
    __global float * nz, 
    __global float * a, 
    __global float * s)
{ 
    int i = get_global_id(0)*3;
    float3 vec1 = (float3)(vx[i], 
              vy[i], vz[i]); 
    float3 vec2 = (float3)(vx[i+1], 
               vy[i+1], vz[i+1]); 
    float3 vec3 = (float3)(vx[i+2], 
              vy[i+2], vz[i+2]); 
    float A = distance(vec1, vec2);
    float B = distance(vec2, vec3);
    float C = distance(vec1, vec3);
    float S = (A + B + C) / 2.0f;
    float sq = sqrt(S * (S - A) * 
            (S - B) * (S - C)) / M_PI;
    s[i] = s[i+1] = s[i+2] = sq;
}

float ElementShadow(float3 v, 
        float rSquared, 
        float3 receiverNormal, 
        float3 emitterNormal, 
        float emitterArea) {
    return (1.0f - rsqrt(emitterArea/
               rSquared + 1.0f)) *
    clamp(dot(emitterNormal, v), 
              0.0f, 1.0f) *
    clamp(4.0f * dot(receiverNormal, 
              v), 0.0f, 1.0f);
}

__kernel void floatAO(
               __global float * vx, 
               __global float * vy, 
               __global float * vz,
	    __global float * nx, 
               __global float * ny, 
               __global float * nz,
	    __global float * a, 
               __global float * s, 
               long from)
{ 
    int i = from + get_global_id(0)*3;
    float res = 0.0f;
    float3 v = (float3)(0, 0, 0);
    float d2 = 0;
    float value = 0;
    for(int j = 0; j<count;j+=3){
        v = (float3)(vx[j], vy[j], vz[j]) - 
               (float3)(vx[i], vy[i], vz[i]);
        d2 = dot(v, v) + 1e-16;                            
        v *= rsqrt(d2);
        value = ElementShadow(v, d2, 
        (float3)(nx[i], ny[i], nz[i]), 
        (float3)(nx[j], ny[j], nz[j]), 
        s[j]);
        res += value;
    }                        
 a[i] = a[i+1] = a[i+2] = 1.0f - res;
}
\end{lstlisting}

\clearpage
\Large{Приложение В} \newline
\begin{table}[h]
\begin{tabular}{|cc|}
00:01:43.419 & \\
00:01:44.241  & \\
00:01:32.125  & \\
00:01:44.158  & \\
00:01:37.423  & \\
\end{tabular}
Среднее: 00:01:40.2732
\end{table}


\begin{table}[h]
\begin{tabular}{|cc|}
00:00:30.205 & \\
00:00:29.944  & \\
00:00:30.072  & \\
00:00:29.832  & \\
00:00:30.341  & \\
\end{tabular}
Среднее: 00:00:30.0788
\end{table}


\begin{table}[h]
\begin{tabular}{|cc|}
00:00:02.527 & \\
00:00:02.502  & \\
00:00:02.509  & \\
00:00:02.549  & \\
00:00:02.511  & \\
\end{tabular}
Среднее: 00:00:02.5196
\end{table}
