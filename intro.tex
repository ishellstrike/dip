\clearpage
\addcontentsline{toc}{section}{\rm{\Large{Введение}}}
\begin{flushleft}\Large{Введение}\end{flushleft}
3D-принтер — это периферийное устройство, использующее метод послойного создания физического объекта по цифровой 3D-модели. В настоящее время это очень быстроразвивающаяся область, имеющая массу применений и позволяющая изготавливать объекты, которые невозможно создать другими способами. Существует несколько технологий, с помошью которых происходит создание итогового объекта

\begin{itemize}
\item Лазерная стереолитография
\item Лазерное сплавлени
\item Ламинирование
\item Застывание материала при охлаждении
\item Полимеризация фотополимерного пластика под действием ультрафиолетовой лампы
\item Склеивание или спекание порошкообразного материала
\end{itemize}

Все варианты печати предусматривают использование некоторой трехменой модели для создания программы печати, например на кафедре математического моделирования ЯРГУ, некоторые модели, печатающиеся там, представляют из себя визаулизацию некоторых математических функций. При задании модели функцией получается модель теоретически неограниченной точности, но зачастую в такой модели образуется большое число полигонов, которые находятся внутри модели и не имеют смысла как для печати, так и для рендеринга. В настоящей работе целью была разработка программы, предназначенной для оптимизации таких моделей т.е. отсечения геометрии, которая находится внутри модели, будем называть такую геометрию внутренней.
