\clearpage
\section{Входные данные}
\subsection{STL}
STL (от англ. stereolithography) -- открытый формат файла используемый для хранения 3D моделей и является основным для большинства 3D принтеров, за исключением внутренних форматов, которые рассматривать не будем, т.к. число этих форматов постоянно увеличивается и, фактически, большинство разработчиков аппаратуры 3D печати создают собственные форматы под свои нужды. Информация в STL может хранится как в текстовом, так и в двоичном виде. Вся геометрия задается в виде наборов точек, составляющих треугольник и соответствующую им нормаль, которая не используется при печати. Этот формат был выбран основным форматом для программы по причине его простоты, достаточности для целей не цветной печати и большой распространенности. Формат используется как для загрузки моделей, так и для сохранения.
\subsection{Obj}
Более продвинутый формат, чем STL, разработанный в Wavefront Technologies. Формат так-же является открытым и очень распространен, поддерживается большинством современных 3D редакторов, но в отличии от STL имеет более расширенные возможности, такие как хранение текстурных координат, нетреугольных полигонов, а так же содержат дополнительные файлы, описывающие материалы, используемые для текстурирования. Используется в проекте для загрузки моделей, выбран по причине простоты и большой распространенности.
